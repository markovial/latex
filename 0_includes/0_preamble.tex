% DOCUMENT CLASS , ENCODING , TITLE -------------------------------------- {{{

\documentclass[a4paper,twocolumn,10pt]{article}
\setlength{\columnsep}{20pt}
%\setlength{\columnseprule}{1pt}
\usepackage[utf8]{inputenc}
\usepackage[english]{babel}

\title{This is title for this document}
\author{John Doestar}
\date{\today}

% }}}
% FONTS ------------------------------------------------------------------ {{{

\usepackage{courier}     % courier font
%\usepackage{fontawesome} % font awesome glyphs
\usepackage{setspace}    % inter line spacing
\singlespacing           % set spacing to single spaced

% \onehalfspacing
% \doublespacing
% \setstretch{1.1}


% }}}
% PAGE FORMAT ------------------------------------------------------------ {{{

\usepackage[document]{ragged2e} % Text alignment package
\usepackage{enumitem}           %
\usepackage{geometry}           %
\geometry{
	a4paper,
	total = {170mm,257mm},
	top=20mm,
	left=20mm,
	right=20mm,
	bottom=20mm
}


\usepackage[switch]{lineno} % switch allows line numbers on alternating sides
%\usepackage[switch,displaymath,mathlines]{lineno}
%\modulolinenumbers[5]
\linenumberfont{\normalfont\large\sffamily}
\renewcommand{\thelinenumber}{\color{gray}\arabic{linenumber}}

\setlength\linenumbersep{0.5cm}


% }}}
% TEXT FORMAT ------------------------------------------------------------ {{{

\usepackage[activate={true,nocompatibility},
final,
tracking=true,
kerning=true,
spacing=true,
factor=1100,
stretch=10,
shrink=10]{microtype}
% prevents a certiain amount of overfull hbox badness
% helps with other margin stuff

\usepackage{verbatim}
\makeatletter
\newcommand{\verbatimfont}[1]{\def\verbatim@font{#1}}%
\makeatother%\verbatimfont{courier}

\usepackage{type1cm}  % Allows font resizing
\usepackage{lettrine} % Allows font calligraphy (enlarge first char)
\renewcommand{\LettrineTextFont}{\rmfamily}

\renewcommand{\baselinestretch}{1} % Line Spacing

%1.0 	single spacing
%1.3 	one-and-a-half spacing
%1.6 	double spacing 

\setlength\parskip{1em} % Space Between Paragraphs
\setlength\parindent{0em} % Indent Spacing 

% }}}
% TITLE FORMAT (TITLESEC) ------------------------------------------------ {{{

\usepackage[export]{adjustbox}
\usepackage{changepage}
\usepackage[compact,explicit]{titlesec} % Allows customization of section head

% compact : reduces spaces before and after sections
% explicit : allows for expicit positioning of title statement with #1

% title_format : command,shape,format,label,sep,before,after
% title-brackets : {}[]{}{}{}{}[] , basically only shape and aftercode have []

% Section Title Settings {{{
\titleformat {\section}
	[hang]
	{\Large\bfseries}
	{}
	{0em}
	{
	\nolinenumbers
	\vspace{-0.5cm}
	\begin{section-box}
		\color{white} \thesection. #1
	\end{section-box}
	}
[
\linenumbers
]
% left before-sep after-sep right-sep
\titlespacing{\section}{0.1cm}{0cm}{0cm}[0cm]

% }}}

% Sub-Section Title Settings {{{
\titleformat {\subsection}
	[hang]
	{\small\bfseries}
	{}
	{0em}
	{
	\nolinenumbers 
	\vspace{-0.5cm} 
	\begin{subsection-box} 
	\begin{subsection-box-2} \end{subsection-box-2} 
	\vspace{-0.5cm} \hspace{0.35cm}
		\color{black} \thesubsection. #1
	\end{subsection-box}
	}
[
\linenumbers
]

% left before-sep after-sep right-sep
\titlespacing*{\subsection}{0cm}{0cm}{0cm}[0em]

% }}}

% Sub-Sub-Section Title Settings {{{
\titleformat {\subsubsection}
	[hang]
	{\small\bfseries}
	{}
	{0em}
	{
	\nolinenumbers 
	\vspace{-0.5cm} 
	\begin{subsubsection-box} 
	\begin{subsubsection-box-2} 
	\begin{subsubsection-box-3}
	\end{subsubsection-box-3} 
	\end{subsubsection-box-2}
	\vspace{-0.4cm} \hspace{0.8cm} 
		\color{black} \thesubsubsection. #1
	\end{subsubsection-box}
	}
[
\linenumbers
]

% left before-sep after-sep right-sep
\titlespacing{\subsubsection}{0cm}{0cm}{0cm}[0em]

% }}}

% }}}
% HYPERLINKS ------------------------------------------------------------- {{{

\usepackage[unicode,bookmarks]{hyperref} % auto hyperlinks toc , refrences
                      % others can be manually specified
					  % unicode + bookmarks for japanese characters

\hypersetup{
colorlinks = true,
linktoc    = all,
citecolor  = purple,
filecolor  = black,
linkcolor  = black,
urlcolor   = black
}


% }}}
% HEADER / FOOTER -------------------------------------------------------- {{{

\usepackage{fancyhdr} % allows for header and footer customizations
\pagestyle{fancy}     %
\fancyhf{}            %

\renewcommand{\headrulewidth}{0.2pt} % draw line at header
\lhead { \textit{ \leftmark } }           % LEFT  : show section name at header
\rhead{\textit{\thepage}}            % RIGHT : Show page number
% \chead{ }

\renewcommand{\footrulewidth}{0pt} % draw line at footer
%\lfoot{\textit{Last Edited : \today}}
%\cfoot{center foot}
%\rfoot{\textit{tiwathia \thepage}}

\pagenumbering{arabic} % Specify type of number characters to use

% }}}
% COLORS ----------------------------------------------------------------- {{{

\usepackage [table]{xcolor}
%\rowcolors{<starting row index>}{<odd row color>}{<even row color>}

% Section Colors {{{

% 464444 is a tint color of 181616
\definecolor {section-bg}         {HTML} {464444}
\definecolor {subsection-bg}      {HTML} {747373}
\definecolor {subsubsection-bg}   {HTML} {8b8a8a}
\definecolor {section-font}       {RGB}  { 0,0,0}
\definecolor {subsection-font}    {RGB}  { 0,0,0}
\definecolor {subsubsection-font} {RGB}  { 0,0,0}

% }}}

% Table Colors {{{

% 404b4b is ananlagous color of 181616
% 534e4e is tint color of 404b4b
\definecolor {table-topic}        {HTML} {747373}
\definecolor {table-subtopic}     {HTML} {8b8a8a}
\definecolor {table-subsubtopic}  {HTML} {a2a1a1}

% e7e7e7 is tint color of 181616
\definecolor {table-alternating-1}    {HTML} {e7e7e7} % 1 = gray
\definecolor {table-alternating-2}    {HTML} {FFFFFF} % 2 = white

\definecolor {cell-lightblue}     {HTML} { b2cce5}
\definecolor {cell-lightgray}     {HTML} { d8deda}
\definecolor {cell-lightorange}   {HTML} { F6C396}
\definecolor {cell-lightred}      {HTML} { F1A099}
\definecolor {cell-lightgreen}    {HTML} { C6DA7F}
\definecolor {cell-lightpurple}   {HTML} { CCB2E5}
\definecolor {cell-lightyellow}   {HTML} { FFF09A}

% }}}

% Tcolorbox Tables {{{


% B/W DEFS

\definecolor {defn-bg}       {HTML} {e7e7e7}
\definecolor {defn-title}    {HTML} {747373}
\definecolor {defn-theword}  {HTML} {a2a1a1}

\definecolor {note-bg}       {HTML} {e7e7e7}
\definecolor {note-theword}  {HTML} {a2a1a1}

\definecolor {table-bg}      {HTML} {FFFFFF}
\definecolor {table-title}   {HTML} {747373}
\definecolor {table-theword} {HTML} {a2a1a1}

\definecolor {image-bg}      {HTML} {FFFFFF} 
\definecolor {image-title}   {HTML} {747373}
\definecolor {image-theword} {HTML} {a2a1a1}

% COLOR DEFS

%\definecolor {defn-bg}       {HTML} {e7e7e7}
%\definecolor {defn-title}    {HTML} {A9A9A9}
%\definecolor {defn-theword}  {HTML} {EDA72D}

%\definecolor {note-bg}       {HTML} {e7e7e7}
%\definecolor {note-theword}  {HTML} {E25A22}

%\definecolor {table-bg}      {HTML} {e7e7e7}
%\definecolor {table-title}   {HTML} {F2F3F4}
%\definecolor {table-theword} {HTML} {9ACD32}

%\definecolor {image-bg}      {HTML} {e7e7e7}
%\definecolor {image-title}   {HTML} {A9A9A9}
%\definecolor {image-theword} {HTML} {0067A5}
% }}}

% Code {{{

\definecolor{code-bg}      {HTML}{fbfbfb}
\definecolor{code-comment} {HTML}{A4E400}
\definecolor{code-keyword} {HTML}{FC1A70}
\definecolor{code-string}  {HTML}{62d8f1}
\definecolor{code-regular} {RGB}{39,40,34}

% }}}

% }}}
% TABLES ----------------------------------------------------------------- {{{

\usepackage{booktabs}    % nicer line drawing with toprule,midrule,hrule
\usepackage{multicol}    % allows column merging in tables
\usepackage{multirow}    % allows row merging in tables
\usepackage{diagbox}     % allows diagonal / angled splitting in table cells
\usepackage{slashbox}    % allows drawing angled slashed in table cells

% was causing issues where equations , matrices , merged cells etc... , were
% appearing white / not displaying / overwriting each other
%\usepackage{tabularx}    % allow tables to stretch to page length

\usepackage{xtab}        % allows page breaking tables inline
\usepackage{makecell}    % allows linebreaking within cells of tables

% longtables tend to break in two column mode
%\usepackage{longtable}  % allows tables to span pages
%\usepackage{ltablex}    % combination of longtable and tabularx


% }}}
% TCOLORBOX -------------------------------------------------------------- {{{

\usepackage[skins,breakable]{tcolorbox}
% skins allows use of enhanced options
% breakable allows breaking boxes between pages

% 	IMAGE TABLES (TCOLORBOX) {{{

\newtcolorbox{image-bg}[2][]{
	enhanced,
	colback           = image-bg,
	colframe          = image-bg,
	fonttitle         = \bfseries,
	width             = \linewidth,
	beforeafter skip  = 0.5cm,
	sharp corners,
%	drop fuzzy shadow = gray,
	boxrule         = 0mm,
	top             = 0mm,
	bottom          = 0mm,
	left            = 0mm,
	right           = 0mm,
	title = #2,#1
}

\newtcolorbox{image-title}[2][]{
	enhanced,
	colback           = image-title,
	colframe          = image-title,
	fonttitle         = \bfseries, 
	width             = \linewidth,
	height            = 0.6cm,
%	drop fuzzy shadow = gray,
	beforeafter skip  = 0pt,
%	grow to left by   = 0.7cm,
	boxrule           = 0mm,
	top               = 0.5mm,
	bottom            = 0mm,
	left              = 1mm,
	right             = 0mm,
	sharp corners,
	title = #2,#1
}

\newtcolorbox{image-theword}{
	enhanced,
	colback           = image-theword,
	colframe          = image-theword,
	fonttitle         = \bfseries,
%	drop fuzzy shadow = gray,
	width             = 3cm,
	height            = 0.5cm,
	beforeafter skip  = 0pt,
%	grow to left by   = 0.7cm,
	boxrule           = 0mm,
	top               = 0.5mm,
	bottom            = 0mm,
	left              = 1mm,
	right             = 0mm,
	sharp corners
}

\newtcolorbox{image-content}{
	enhanced,
	colback         = image-bg,
	colframe        = image-bg,
	fonttitle       = \bfseries,
%	enlarge top by  = -0.5cm,
%	enlarge right by = 5cm,
	width           = \linewidth,
	boxrule         = 0mm,
	top             = 0mm,
	bottom          = 0mm,
	left            = 0mm,
	right           = 0mm,
%	show bounding box
}


\newtcolorbox{image-caption}{
	enhanced,
	colback           = image-theword,
	colframe          = image-theword,
	fonttitle         = \bfseries, 
	width             = 0.18\linewidth,
	height            = 0.5cm,
	beforeafter skip  = 0pt,
%	grow to left by   = 0.7cm,
	boxrule           = 0mm,
	top               = 0.5mm,
	bottom            = 0mm,
	left              = 1mm,
	right             = 0mm,
	sharp corners
}


% }}}

% 	REGULAR TABLES (TCOLORBOX) {{{

\newtcolorbox{table-bg}[2][]{
	enhanced,
%	float,
%	breakable,
	colback           = table-bg,
	colframe          = table-bg,
	fonttitle         = \bfseries,
%	width             = 0.98\linewidth,
	beforeafter skip  = 0cm,
%	drop fuzzy shadow = gray,
	boxrule         = 0mm,
	top             = 0mm,
	bottom          = 0mm,
	left            = 0mm,
	right           = 0mm,
	sharp corners,
	title = #2,#1
}

\newtcolorbox{table-theword}{
	enhanced,
	colback           = table-theword,
	colframe          = table-theword,
	fonttitle         = \bfseries,
%	drop fuzzy shadow = gray,
	width             = 2.2cm,
	height            = 0.5cm,
	beforeafter skip  = 0pt,
%	grow to left by   = 0.7cm,
	boxrule           = 0mm,
	top               = 0.5mm,
	bottom            = 0mm,
	left              = 1mm,
	right             = 0mm,
	sharp corners,
}

\newtcolorbox{table-title}[2][]{
	enhanced,
	colback           = table-title,
	colframe          = table-title,
	fonttitle         = \bfseries,
	width             = 0.8\linewidth,
	height            = 0.6cm,
%	width = 0.5\linewidth,
%	drop fuzzy shadow = gray,
	beforeafter skip  = 0pt,
%	grow to left by   = 0.7cm,
	boxrule           = 0mm,
	top               = 0.5mm,
	bottom            = 0mm,
	left              = 1mm,
	right             = 0mm,
	sharp corners,
	title = #2,#1
}


\newtcolorbox{table-content}{
	enhanced,
	colback         = table-bg,
	colframe        = table-bg,
	fonttitle       = \bfseries,
	before skip = 0.5cm,
%	enlarge top by  = -0.5cm,
%	enlarge right by = 5cm,
	width           = \linewidth,
	boxrule         = 0mm,
	top             = 2mm,
	bottom          = 0mm,
	left            = 0mm,
	right           = 0mm 
}

% }}}

% 	DEFINITION TABLE (TCOLORBOX) {{{

% \newtcbox[init options]{name}[number][default]{options}

\newtcolorbox{defn-bg}{
	enhanced,
	colback           = defn-bg,
	colframe          = defn-bg,
	fonttitle         = \bfseries,
%	drop fuzzy shadow = gray,
	width             = \linewidth,
	beforeafter skip  = 0.3cm, 
	boxrule           = 0mm,
	top               = 0mm,
	bottom            = 0mm,
	left              = 0mm,
	right             = 0mm,
%	sharp corners
	arc is angular,
}

\newtcolorbox{defn-theword}{
	enhanced,
	colback           = defn-theword,
	colframe          = defn-theword,
	fonttitle         = \bfseries,
	width             = 3.3cm,
	height            = 0.5cm,
	beforeafter skip  = 0pt,
	boxrule           = 0mm,
	top               = 0.5mm,
	bottom            = 0mm,
	left              = 1mm,
	right             = 0mm,
	drop fuzzy shadow = gray,
	% comment following line to align
	grow to left by   = 0.3cm,
	sharp corners
}

\newtcolorbox{defn-title}[2][]{
	enhanced,
	colback           = defn-title,
	colframe          = defn-title,
	fonttitle         = \bfseries,
	width             = 0.5\linewidth,
	height            = 0.6cm,
	beforeafter skip  = 0pt,
	boxrule           = 0mm,
	top               = 0.5mm,
	bottom            = 0mm,
	left              = 1mm,
	right             = 0mm,
	drop fuzzy shadow = gray,
	% comment following line to align
	grow to left by   = 0.3cm,
	sharp corners,
	title = #2,#1
}

\newtcolorbox{defn-content}{
	enhanced,
	colback         = defn-bg,
	colframe        = defn-bg,
	fonttitle       = \bfseries,
	beforeafter skip = 0.5cm,
%	enlarge top by  = -0.5cm,
%	enlarge right by = 5cm,
	width           = \linewidth,
	boxrule         = 2.5mm,
	top             = 2mm,
	bottom          = 0mm,
	left            = 0mm,
	right           = 0mm 
}

% }}}

% 	NOTE TABLE (TCOLORBOX) {{{

\newtcolorbox{note-bg}{
	enhanced,
	colback           = note-bg,
	colframe          = note-bg,
	fonttitle         = \bfseries,
%	drop fuzzy shadow = gray,
	width             = 0.95\linewidth,
	before skip       = 0cm,
	after skip        = 0cm,
	boxrule           = 0mm,
	top               = 0mm,
	bottom            = 0mm,
	left              = 0mm,
	right             = 0mm,
%	sharp corners, 
	arc is angular,
}

\newtcolorbox{note-theword}{
	enhanced,
	colback           = note-theword,
	colframe          = note-theword,
	fonttitle         = \bfseries,
%	drop fuzzy shadow = gray,
	width             = 1.4cm,
	height            = 0.5cm,
	beforeafter skip  = 0pt,
	boxrule           = 0mm,
	top               = 0.5mm,
	bottom            = 0mm,
	left              = 1mm,
	right             = 0mm,
%	grow to left by   = 0.7cm,
	sharp corners,
}

\newtcolorbox{note-content}{
	enhanced,
	colback         = note-bg,
	colframe        = note-bg,
	fonttitle       = \bfseries,
	boxrule         = 0mm,
	top             = 2mm,
	bottom          = 0mm,
	left            = 0mm,
	right           = 0mm 
%	enlarge top by  = -0.6cm,
%	enlarge left by = 1.5cm,
%	width           = 13cm,
}


% }}}

% 	SECTION TITLES (TCOLORBOX) {{{

\newtcolorbox{section-box}{
	enhanced,
	colback          = section-bg,
	colframe         = section-bg,
	fonttitle        = \bfseries,
	width            = \linewidth,
	height           = 1cm,
	beforeafter skip = 0pt, 
	sharp corners 
}

\newtcolorbox{subsection-box}{
	enhanced,
	colback          = subsection-bg,
	colframe         = subsection-bg,
	fonttitle        = \bfseries,
	width            = \linewidth,
	height           = 0.8cm,
	beforeafter skip = 0pt,
	boxrule          = 0mm, 
	top               = -1mm,
	bottom            = 0mm,
	left              = -1mm,
	right             = 0mm,
	sharp corners
}

\newtcolorbox{subsubsection-box}{
	enhanced,
	colback   = subsubsection-bg,
	colframe  = subsubsection-bg,
	fonttitle = \bfseries,
	width     = \linewidth,
	height = 0.6cm,
	sharp corners,
	beforeafter skip  = 0pt,
	boxrule           = 0mm,
	top               = -1mm,
	bottom            = 0mm,
	left              = -1mm,
	right             = 0mm
}

% }}}

% 	SECTION TITLES 2 (TCOLORBOX) {{{

\newtcolorbox{subsection-box-2}{
	enhanced,
	colback          = section-bg,
	colframe         = section-bg,
	fonttitle        = \bfseries,
	width            = 0.35cm,
	height           = 0.8cm,
	beforeafter skip = 0pt, 
	top               = 0mm,
	bottom            = 0mm,
	left              = 0mm,
	right             = 0mm,
	sharp corners 
}

\newtcolorbox{subsubsection-box-2}{
	enhanced,
	colback          = section-bg,
	colframe         = section-bg,
	fonttitle        = \bfseries,
	width            = 0.35cm,
	height           = 0.6cm,
	beforeafter skip = 0pt, 
	top               = -1.5mm,
	bottom            = 0mm,
	left              = 2mm,
	right             = 0mm,
	sharp corners 
}


\newtcolorbox{subsubsection-box-3}{
	enhanced,
	colback          = subsection-bg,
	colframe         = subsection-bg,
	fonttitle        = \bfseries,
	width            = 0.35cm,
	height           = 0.6cm,
	beforeafter skip = 0pt,
	boxrule          = 0mm, 
	top               = 0mm,
	bottom            = 0mm,
	left              = 0mm,
	right             = 0mm,
	sharp corners
}

% }}}

% }}}
% IMAGES ----------------------------------------------------------------- {{{

\usepackage{wrapfig, graphicx}
%\usepackage{graphicx}   %
\usepackage{subcaption} %
%\usepackage{wrapfig}    % wrap images around text
\usepackage{capt-of}    % define captions independent of figures
\usepackage{float}
\usepackage{varwidth}
\graphicspath{{2_images/}} % define folder path for images

% }}}
% CODE / CODE DISPLAY ---------------------------------------------------- {{{

\usepackage{listings}

\lstset{
basicstyle       = \footnotesize\ttfamily,
backgroundcolor  = \color{code-bg},
commentstyle     = \color{code-comment}, % comment style
keywordstyle     = \color{code-keyword},  % keyword style
stringstyle      = \color{code-string},  % string literal style
rulecolor        = \color{black},        % if not set, the frame-color may be changed on line-breaks
frame            = single,               % adds a frame around the code
basicstyle       = \footnotesize,        % the size of the fonts that are used for the code
keepspaces       = true,                 % keeps spaces in text, useful for keeping indentation of code
tabsize          = 2,                    % sets default tabsize
breaklines       = true,                 % sets automatic line breaking
captionpos       = b,                    % sets the caption-position to bottom
%numbers          = left,
%numberstyle      = \tiny,
%numbersep        = 10pt,
frame            = tb, % top bottom
columns          = fixed,
showstringspaces = false,
showtabs         = false,
keepspaces,
% escapeinside={\%*}{*)},  % if you want to add LaTeX within your code
framextopmargin=10pt,    % margin for the top background border
framexbottommargin=10pt, % margin for the bot background border
framexleftmargin=0pt,    % margin for the left background border
framexrightmargin=0pt    % margin for the right background border
}


% }}}
% MATH / GRAPHING  ------------------------------------------------------- {{{

\usepackage{amsmath}  % basic math package
\usepackage{amssymb}  % allows more math symbols
\usepackage{amsthm}   % allows custom therorem,defn,corll etc... definitions
\usepackage{mathrsfs} % some new fonts for math mode
\usepackage{cancel}   % allows me to cancel , and cancelto text 

\newtheorem{mydef}{DEFINITION}[section] % to get auto numbering
\newtheorem{myfigure}{FIGURE}[section]  % to get auto numbering
\newtheorem{mytable}{TABLE}[section]    % to get auto numbering

% tex.stackexchange.com/questions/461186/how-to-use-lineno-with-amsmath-align
% amsmath should come after lineno for auto math line numbering
% linenumbers for amsmath (lineno+ams) ,
% honestly i can live without it ... , it creates more problems than its worth




% TIKZ -------------------------------------------------------------------- {{{
\usepackage{tikz}
%\documentclass[tikz, border=1mm]{standalone} % only tikz image document
\usetikzlibrary{shapes.geometric}
\usetikzlibrary{shapes.arrows}
\usetikzlibrary{positioning}
\usetikzlibrary{backgrounds}
\usetikzlibrary{decorations.pathreplacing,angles,quotes}

\usepackage{pgfplots}
\pgfplotsset{compat=newest}
\usepgfplotslibrary{fillbetween}
\usepackage{array}


% }}}

% }}}
% BIBLIOGRAPHY / REFERENCES / FOOTNOTES ---------------------------------- {{{

\usepackage[nottoc,numbib]{tocbibind} % to show references line in toc
\usepackage{natbib}                   % natbib bibliography manager
\bibliographystyle{rusnat}            % natbib bibliography style
\setcitestyle{notesep={; },round,aysep={},yysep={;}} % natbib inline citation style

%\usepackage[superscript,biblabel]{cite}

%\usepackage{cleveref} % enhances latex referencing features
                      % stuff like page refrences , formats etc
% it might help get rid of all those custom secrefs we have going on

\renewcommand{\thefootnote}{\roman{footnote}} % footnote style

% force footnotes to bottom of page
\def\footnoterule{\vfill % added this
   \kern-3pt\hrule width 2truein \kern 2.6pt} % the \hrule is .4pt high

% Dont forget to generate the .bbl and .blg files using :
% pdflatex document.tex
% bibtex document
% pdflatex document.tex


% }}}
% TABLE OF CONTENTS ------------------------------------------------------ {{{

\usepackage{titletoc}
% margin from RHS
%\contentsmargin{1cm}

%\dottedcontents {section}[left]{above}{label-width}{leader-width}

%\dottedcontents{section}[1.8cm]{\bfseries}{3.2em}{1pc}
%\dottedcontents{subsection}[1.8cm]{}{3.2em}{1pc}
%\dottedcontents{subsubsection}[1.8cm]{}{2.8em}{1pc}


% }}}
% CUSTOM COMMANDS -------------------------------------------------------- {{{

% New Commands : Sectioning {{{

\newcommand{ \sectionend}
{
\nolinenumbers
\begin{center} 
{\large \textbf{-§-}  }
\end{center} 

% was causing problems , i havent figured out the cause
%\justify

\linenumbers
\pagebreak
}

\newcommand{ \subsectionend}
{
\nolinenumbers 

%\begin{center} 
%\rule{0.7em}{0.7em} \rule{0.7em}{0.7em}
%\end{center} 

\linenumbers
}

\newcommand{ \subsubsectionend}
{
\nolinenumbers

%\begin{center} 
%\rule{0.7em}{0.7em}
%\end{center} 

\linenumbers
}

% }}}

% New Commands : Symbols {{{

\newcommand{ \bulletpoint}
{ $\bullet$  }

% }}}

% New Commands : Tcolorbox {{{

%	General {{{

\newcommand { \tcolorboxstart }
{
	\nolinenumbers
	\vspace{0.2cm}
	\centering
}

\newcommand{ \tcolorboxend}
{
	\justifying
	\vspace{0.2cm}
	\linenumbers
}

% }}}

%	Definition {{{

\newcommand{ \tcolorboxdefinition}[3]
{

\tcolorboxstart
\begin{defn-bg}

	\begin{defn-title}[width=7cm]{}
	{
		\normalsize \textbf{#1}
	}
	\end{defn-title}

	\begin{defn-theword}
	{
		\footnotesize
		\begin{mydef} #2
%		\label{def:{#2}}
		\end{mydef}
	}
	\end{defn-theword}

	\begin{defn-content}
		\justify
		\vspace{-0.6cm}
		#3
	\end{defn-content}

\end{defn-bg}
\tcolorboxend
}

% }}}

%	Note {{{

\newcommand{ \tcolorboxnote}[1]
{

\tcolorboxstart
\begin{note-bg}

	\begin{note-theword}
		{\footnotesize \textbf{NOTE} }
	\end{note-theword}

	\begin{note-content} \justifying

		#1

	\end{note-content}

\end{note-bg}
\tcolorboxend
}




% }}}

%	Table {{{


\newcommand{ \tcolorboxtable}[5]
{
\tcolorboxstart
\begin{table-bg}#3{}

	\begin{table-title}[width=\linewidth]{}
		\captionsetup{labelformat=empty}
		\captionof{table}{ \textbf{#1} }
	\end{table-title}

	\begin{table-theword}
		\footnotesize
		\begin{mytable}
		#2
		\end{mytable}
	\end{table-theword}

	\begin{table-content}
	\vspace{-0.5cm}
	#4
	\end{table-content}

\end{table-bg}
\tcolorboxend
}

% }}}

%	Image {{{

\newcommand{ \tcolorboxfigure}[4]
{
\tcolorboxstart
\begin{image-bg}[width=\linewidth]{}

	\begin{image-title}[width=\linewidth]{}
		\captionsetup{labelformat=empty}
		\captionof{figure}{#1 \cite{#4}}
	\end{image-title}

	\begin{image-theword} 
	\footnotesize 
		\begin{myfigure}
		#2
		\end{myfigure}
	\end{image-theword}

	\begin{image-content}
		\includegraphics[width=\linewidth]{#3} 
	\end{image-content}



\end{image-bg}
\tcolorboxend
}

% }}}

% }}}

% New Commands : Tables {{{



\newcommand{ \tabulartable }[4]
{
	\nolinenumbers

	% emerson.emory.edu/services/latex/latex_69.html
	% \begin{tabular*}{width}[pos]{cols}
	\begin{tabular*}{#1}[#2]{@{\extracolsep{\fill}} #3 @{}}

		#4

	\end{tabular*}

	\linenumbers
}

\newcommand{ \alternatingtabulartable }[2]
{
	\nolinenumbers

	\rowcolors{2}{table-alternating-1}{table-alternating-2}
	\begin{tabular*}{\columnwidth}{@{\extracolsep{\fill}} #1 @{}}

		#2

	\end{tabular*}

	\linenumbers
}

\newcommand{ \alternatingtabularxtable}[3]
{

	\nolinenumbers

	\rowcolors{2}{table-alternating-1}{table-alternating-2}
	\begin{tabularx}{#1}{#2}
		#3
	\end{tabularx}

	\linenumbers

}

\newcommand{ \tabularxtable}[3]
{
	\nolinenumbers

	\rowcolors{2}{table-alternating-2}{table-alternating-2}
	\begin{tabularx}{#1}{#2}
		#3
	\end{tabularx}

	\linenumbers
}

\newcommand{ \alternatingxtabulartable}[2]
{
	\nolinenumbers

	\rowcolors{2}{table-alternating-1}{table-alternating-2}
	\begin{xtabular}{#1}
		#2
	\end{xtabular}

	\vspace{0.25cm}
	\linenumbers
}

\newcommand{ \xtabulartable}[2]
{
	\nolinenumbers

	\rowcolors{2}{table-alternating-2}{table-alternating-2}
	\begin{xtabular}{#1}
		#2
	\end{xtabular}

	\vspace{0.25cm}
	\linenumbers
}

\newcommand{ \twocolumnxtabulartable}[2]
{
	\nolinenumbers

	\rowcolors{2}{table-alternating-2}{table-alternating-2}
	\begin{xtabular}{#1}
		#2
	\end{xtabular}

	\vspace{0.25cm}
	\linenumbers
}

\newcommand{ \pagewidetable}[3]
{

	\nolinenumbers

	\begin{table*}[ht]
	\caption{#1}
	\centering
	\begin{tabular} {#2}
		#3
	\end{tabular}
	\end{table*}

	\linenumbers
}
% }}}

% New Commands : References {{{

\newcommand{ \refsec}[1]
{
	\hyperref[sec:#1]
	{
		(\textsection~\ref{sec:#1})
	}
}

\newcommand{ \refssec}[1]
{
	\hyperref[sec:#1]
	{
		(\textsection~\ref{ssec:#1})
	}
}

\newcommand{ \refsssec}[1]
{
	\hyperref[sec:#1]
	{
		(\textsection~\ref{sssec:#1})
	}
}

\newcommand{ \refdef}[1]
{
	\hyperref[def:#1]
	{
		(Def.~\ref{def:#1})
	}
}

\newcommand{ \reffig}[1]
{
	\hyperref[fig:#1]
	{
		(Fig.~\ref{fig:#1})
	}
}

\newcommand{ \reftable}[1]
{
	\hyperref[table:#1]
	{
		(Table.~\ref{table:#1})
	}
}
% }}}

% }}}
% TESTING ---------------------------------------------------------------- {{{ 

\usepackage{lipsum}    % generates filler text
\usepackage{blindtext} % generates non-latin filler text

%\usepackage{showframe} % allows showing frames within which stuff is printed
%\renewcommand*\ShowFrameColor{\color{red}}


% }}}
% JAPANESE --------------------------------------------------------------- {{{

\usepackage[whole]{bxcjkjatype} % Allows Japanese Characters , CJK incompatible with pdflatex
\usepackage{ruby}               % used for kanji furigana placement
                                % \ruby{学}{がく}

\renewcommand\rubysep{0ex}      % Kanji and furigana placement location
\let\temp\ruby                  %
\renewcommand\ruby[2]{          % Kanji and furigana size / spacing / color
	\temp
	{\fontsize{10}{10}#1 }
	{\fontsize{6}{6}\textcolor{gray}{#2}}
}
% }}}
% KOREAN {{{
\usepackage{CJKutf8}

% text has to inside block :
%\begin{CJK}{UTF8}{}
%\end{CJK}


% }}}


% harpoon symbol  {{{

%\makeatletter
%% \overarrow@ and \arrowfill@ are defined in amsmath
%\newcommand*{\harpoon}{\mathpalette{\overarrow@\rightharpoonupfill@}}
%\newcommand*{\rightharpoonupfill@}{\arrowfill@\relbar\relbar\rightharpoonup}
%\makeatother

%\DeclareFontFamily{U}{matha}{\hyphenchar\font45}
%\DeclareFontShape{U}{matha}{m}{n}{
      %<5> <6> <7> <8> <9> <10> gen * matha
      %<10.95> matha10 <12> <14.4> <17.28> <20.74> <24.88> matha12
      %}{}
%\DeclareSymbolFont{matha}{U}{matha}{m}{n}
%\DeclareMathSymbol{\varrightharpoonup}{3}{matha}{"E1}



% }}}
